\begin{resumo}[Abstract]
 \begin{otherlanguage*}{english}
Research on executive-legislative relations in presidential systems have emphasized how presidents use cabinet appointments to form and manage government coalitions in the absence of majority legislative support. Yet not all coalitions look alike, as some are bigger and, consequently, more prone to agency and coordination problems than others. But what shapes presidents' decision to include more parties in their coalitions? While several hypotheses exist in the literature, few have been tested in a systematic fashion, none focusing on why surplus coalitions form. This paper intends to fill this gap by examining an original time-series cross-sectional dataset comprising 168 unique coalitions in all 18 Latin American presidential countries since 1979. In particular, I find that strong and effective assemblies, presidents with ample legislative powers and highly fragmentated party systems are most likely to generate oversized coalitions in different model specifications.

\vspace{\onelineskip}
\noindent
\textbf{Key-words}: Government coalitions;  Presidentialism; Executive-Legislative relations.
 \end{otherlanguage*}
\end{resumo}