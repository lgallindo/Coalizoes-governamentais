\section{Conclusão}
\label{sec:conclusao}

Das decisões que um presidente toma ao formar um gabinete de coalizão, a de quantos partidos incluir nele é uma das mais importantes. Com um número suficiente de partidos para obter mais de 50\% das cadeiras do congresso, o governo pode contornar problemas de paralisia decisória e instabilidade democrática que supostamente deveriam surgir em regimes com separação de poderes. Contudo, presidentes muitas vezes incluem mais partidos em seus gabinetes, e as razões que justificam esta decisão não são óbvias. Com base em dados dos 18 países presidencialistas da América Latina entre 1979 e 2012, este artigo testou três hipóteses principais para explicar o surgimento deste tipo de coalizão. Particularmente, a análise apontou que congressos fortes, presidentes com maiores poderes legislativos e fragmentação partidária elevada aumentam a probabilidade de que estas coalizões sobredimensionadas surjam.

Em primeiro lugar, legislativos fortes incentivam presidentes a formar coalizões sobredimensionadas porque podem tanto dificultar a implementação da agenda presidencial quanto fiscalizar o executivo. Como o desempenho de um presidente é avaliado em grande medida pela capacidade de introduzir mudanças no \textit{status quo} e de sua imagem como gestor, controlar a agenda legislativa e os principais espaços de fiscalização no congresso são fundamentais para o sucesso presidencial. Se o legislativo for bicameral e a composição partidária não coincidir entre as casas, aprovar uma agenda lesgislativa torna-se mais difícil; além disso, em congressos que estimulam a especialização legislativa e o desenvolvimento de carreiras longas, além do monitoramento do executivo e das contas públicas, presidentes têm maiores fontes potenciais de constrangimentos. Deste modo, quanto mais forte for congresso, maiores os incentivos do presidente para ampliar a sua base de sustentação. Particularmente, os resultados sugerem que, de fato, congressos fortes e efetivos aumentam de forma não desprezível a probabilidade de surgir uma coalizão \textit{surplus}, o que vai ao encontro de outros estudos que destacam a importância do legislativo da definição das estratégias presidenciais (ALEMAN e TSEBELIS, \citeyear{aleman2011}; COX e MORGENSTER, \citeyear{cox2001}).

Em segundo lugar, a análise também indicou que o número de partidos e a polarização ideológica no congresso importam na explicação da ocorrência de coalizões sobredimensionadas. Conforme argumentado, maior fragmentação partidária aumenta a oferta de potenciais parceiros à disposição de um presidente e possibilita a estes tornarem-se menos vulneráveis às pressões de membros da coalizão através da inclusão de mais partidos no gabinete. Em outras palavras, quanto maior a influência de cada partido membro do gabinete no resultado das votações em plenário, maior o incentivo para formar coalizões sobredimensionadas. Esta hipótese é corroborada pelos resultados e explica até um terço da probabilidade de ocorrência do fenômeno, todo o resto constante. A polarização ideológica, por seu turno, também apresenta impacto, mas na direção contrária: maior polarização reduz a probabilidade de ocorrência de coalizões \textit{surplus}. Quanto mais distante ideologicamente os partidos num congresso, portanto, mais difícil é formar uma coalizão que incorpore muitos deles.

O único resultado que contraria as expectativas iniciais é do efeito positivo que maior poder presidencial de legislar exerce sobre a probabilidade de surgir uma coalizão sobredimensionada. De acordo com grande parte da literatura comparada sobre o tema, presidentes que dispõem de certos poderes legislativos, como expedir decretos legislativos e controlar o processo orçamentário, teriam incentivos para legislar unilateralmente quando os seus interesses divergem dos do congresso. Entretanto, mesmo controlando o extremismo ideológico e a percentagem de cadeiras do partido do presidente, maiores poderes legislativos parecem não diminuir a probabilidade de vermos um gabinete sobredimensionado, mas antes a \textit{aumentar} substantivamente. Uma explicação para isso não é óbvia, e certamente isso demanda maiores investigações. De qualquer forma, como o estudo de Pereira et al. (\citeyear{pereira2005}) sobre o uso de decretos legislativos no Brasil sugere, a teoria da ação presidencial unilateral pode ser complementada por outra, a da delegação, segundo a qual presidentes fortes se valem de seus poderes legislativos para coordenar o processo decisório e adquirir informação no lugar do congresso, e não contra ele. Sendo este o caso, presidentes com maiores poderes legislativos seriam justamente os com maior capacidade de gerenciar coalizões com muitos membros.

Por fim, este artigo também procurou chamar a atenção para a variação nos tipos de coalizão sobredimensionada. Um sistema partidário com muitos partidos torna possível a existência de coalizões com diversos partidos excedentes, enquanto, em outros, a disponibilidade de partidos não permite a inclusão de membros adicionais; ademais, estes diferentes padrões tendem a se manter em determinados países. Ainda que este artigo não forneça uma explicação para estas variações, algumas vias de investigação podem ser pensadas a partir daqui. Através de uma análise de série temporal com apenas um ou dois países, seria possível examinar mais especificamente quais fatores influenciam o tamanho das coalizões. Países como Bolívia e Peru seriam indicados, neste caso, porque possuem maior variação nos tipos de coalizão desde a redemocratização; pela disponibilidade de dados, por outro lado, o caso brasileiro poderia ser estudado com o uso do número de partidos adicionais na coalizão como variável dependente, o que permitiria, também, identificar a razão para o aumento contínuo destes nos últimos anos.