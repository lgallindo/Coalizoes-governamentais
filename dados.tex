\subsection{Dados e método}
\label{sec:dados}

Examino os determinantes das coalizões governamentais sobredimensionadas com um banco de dados contendo informações sobre 168 coalizões governamentais únicas\footnote{Coalizões únicas são todas aquelas em que a composição partidária de um gabinete se manteve constante -- o que evidentemente engloba como coalizão gabinetes unipartidários.} em todos os 18 países presidencialistas da América Latina\footnote{Estes países são: Argentina (1983-2012), Bolívia (1982-2012), Brasil (1985-2012), Chile (1990-2012), Colômbia (1978-2012), Costa Rica (1978-2012), El Salvador (1994-2012), Equador (1979-1995, 1997-2005), Guatemala (1996-2011), Honduras (1982-2008, 2010-2012), México (1988-2011), Nicarágua (1997-2006), Panamá (1991-2008), Paraguai (1993-1997, 1999-2011), Peru (1980-2008), Rep. Dominicana (1996-2009), Uruguai (1985-2012) e Venezuela (1979-2001). As fontes utilizas para compilar todas as variáveis utilizadas são descritas no Apêndice A.}. Os dados cobrem o período que vai de 1979 até 2012, totalizando 439 observações no formato país-ano. Além da vantagem óbvia do aumento na amostra e na variação nos preditores, a inclusão dos países da América Central -- frequentemente ignorados pela literatura sobre o tema -- também me permite testar hipóteses elaboradas para os países do Cone Sul num contexto mais amplo.

\subsubsection{Variável dependente: coalizões sobredimensionadas}

Antes de classificar as coalizões de acordo com seus tamanhos, é necessário definir governo de coalizão. Seguindo outros estudos (e. g., AMORIM NETO, \citeyear{neto2006}; FIGUEIREDO et al., \citeyear{figueiredo2012}; MARTINEZ-GALLARDO, \citeyear{martinez2012}), o critério aqui utilizado para identificá-los é a filiação partidária dos ministros das principais pastas de cada país. A principal vantagem deste é tornar possível a repetição da coleta de dados, já que não deixa margem a erros de imputação. Por outro lado, o procedimento possui um maior problema: o suporte legislativo dado ao governo por um partido que possui membros no gabinete não é automático. Ministros nem sempre são recrutados pelas conexões ou influência legislativa que possuem; em outros casos, não são reconhecidos pelos seus partidos como representantes legítimos. Sem levar isso em conta, falsos positivos, isto é, governos erroneamente classificados como de coalizão, acabariam sendo compilados. Para contornar este problema, consultei especialistas em alguns dos países incluídos na amostra e outras três bases de dados sobre governos de coalizão para checar cada observação\footnote{Os dados que serviram de base me foram gentilmente cedidos por Octavio Amorim Neto e Cecília Martinez-Gallardo. Após compilar os casos remanescentes a partir das fontes descritas no Apêndice A, chequei parte dos resultados com os dados de Cheibub (\citeyear{cheibub2007}), Chasquetti (\citeyear{chasquetti2001}) e Saez e Montero (\citeyear{alcantara2008}). De todo o modo, para alguns países, sobretudo dos Andes e da América Central, os resultados ou não coincidiram ou não eram precisos. Nestes casos, pedi para os seguintes especialistas confirmarem a composição dos gabinetes e indicar se eram ou não de coalizão: Ivana Grace Deheza (Bolívia); Felipe Botero (Colômbia); Evelyn Villareal Fernandez e Jorge Vargas Cullel (Costa Rica); Alvaro Artiga e Nivaria Ortega (El Salvador); Eduardo Dargent e Paula Munoz Chirinos (Peru); e Rosario Espinal (Rep. Dominicana).}. Sempre que os dados originais divergiram dos de outras bases e estudos, a classificação final dos especialistas foi adotada. Feito isto, o número de partidos e cadeiras de cada coalizão foram compilados a partir da composição partidária corrigida de cada gabinete.

A existência de coalizões sobredimensionadas foi mensurada de duas formas. Primeiro, através de uma variável dicotômica que assume o valor 1 sempre que ao menos um dos membros da coalizão pudesse ser removido sem perda de maioria no congresso ou, em países bicamerais, na câmara baixa. Esta é a operacionalização convencional de coalizões \textit{surplus} (e. g., CROMBEZ, \citeyear{crombez1996}; MARTINEZ-GALLARDO, \citeyear{martinez2012}; VOLDEN e CARRUBBA, \citeyear{volden2004}). Para captar maior variação e testar a robustez dos resultados, entretanto, uma segunda variável conta o número de partidos que poderiam ser removidos sem perda de \textit{status} majoritário\footnote{Obviamente, as maiorias requeridas para aprovação de leis ordinárias varia entre os países incluídos na amostra: em alguns, a maioria absoluta dos membros presentes é necessária; em outros, a maioria simples dos membros presentes (MONTERO, \citeyear{montero2013}). De qualquer forma, se em N votações o número de presentes for estipulado de maneira aleatória, quanto maior for esse N maior será o número de vitorias do partido/coalizão com maioria legislativa; e, em última instância, qualquer votação poderá ser vencida desde que o total de parlamentares do lado majoritário seja mobilizado.}.

\subsubsection{Variáveis independentes}

Como dito anteriormente, as três principais hipóteses discutidas destacam o arranjo institucional e o contexto que estruturam a relação executivo-legislativo como os principais fatores que a explicam a formação de coalizões sobredimensionadas. Especificamente, elas preveem que a existência destas coalizões varia de acordo com os poderes legislativos do presidente, a força do congresso e demais fatores que influenciam a quantidade e as preferências de cada partido.

Para avaliar o efeito dos poderes legislativos dos presidentes, utilizo o índice desenvolvido por Negretto (\citeyear{negretto2013}) feito a partir de um modelo de variável latente estandardizado com valores de 0 a 100. O índice resume 14 indicadores categóricos de poder presidencial, como poder de veto e possibilidade de iniciar referendos. Quanto maiores os valores neste índice, mais forte é o presidente e, portanto, menos incentivos ele teria para formar coalizões \textit{surplus}. De qualquer modo, pode-se alegar que nem todos os poderes presidenciais geram incentivos para a formação de coalizões. Por isso, também utilizo outras três alternativas para mensurar aspectos diretamente relevantes dos poderes presidenciais. A primeira é um indicador da disponibilidade da prerrogativa de veto presidencial, que assume o valor de 1 quando presidente pode vetar parcialmente um projeto, 0.5 quando pode vetar totalmente um projeto e 0 quando não dispõe de poder algum; a segunda é um indicador do poder presidencial de expedir decretos legislativos que assume um valor máximo de 1 quando o presidente que também varia de 0, na ausência desta prerrogativa, a 1. Ambas as variáveis foram tomadas de Negretto (\citeyear{negretto2013}). Adicionalmente, também uso uma \textit{dummy}, tomada de Cheibub (\citeyear{cheibub2007}), para mensurar o poder orçamentário do presidente que assume o valor de 1 quando este domina o processo, isto é, quando é capaz de iniciar propostas orçamentárias sem sofrer vetos do congresso ou quando a falha em aprovar uma proposta alternativa beneficia o presidente.

Quanto à força do congresso, isto é, à capacidade do legislativo de impedir mudanças no \textit{status quo} e de fiscalizar e checar o executivo, não existe solução clara. Estudos comparados sobre o legislativo na América Latina são poucos, bem como a disponibilidade de índices para mensurar e classificar suas capacidades (ALEMAN, \citeyear{aleman2013}). Para capturar este fator, assim, recorro ao \textit{polcon3}, uma índice que atribui valores entre 0, quando mudanças no \textit{status quo} não sofrem qualquer tipo de impedimento, e  1, quando o \textit{winset} do \textit{status quo} é vazio (HENISZ, \citeyear{henisz2002}). Ele é especialmente adequado porque leva em consideração o número de casas legislativas, a fragmentação e a coincidência na distribuição partidária em cada uma delas ao atribuir os \textit{scores}. O ponto negativo desta variável, contudo, é o de não considerar a capacidade de fiscalização do congresso, que também pode ser usada para induzir a cooperação do executivo. Para mensurar isto, incluo alternativamente uma variável sobre a efetividade do legislativo calculada a partir da resposta de altos executivos no setor privado à pergunta “Quão efetivo é o parlamento/congresso nacional na produção legiferante e fiscalização?”, feita pelo \textit{Global Competitiveness Report} do \textit{World Economic Forum} (STEIN et al., \citeyear{stein2006}). Os valores vão de 1, “muito ineficiente”, a 7, “muito eficiente”. Apesar de não estar livre de problemas, esta é a principal solução em estudos que consideram o poder do legislativo (e. g., ALEMAN e TSEBELIS, \citeyear{aleman2011}; MARTINEZ-GALLARDO, \citeyear{martinez2012}). 

As variáveis contextuais utilizadas são três. A primeira delas é o Número Efetivo de Partidos Parlamentares, que é igual ao somatório das percentagem de cadeiras de cada partido ao quadrado elevado na menos um, i. e., $NEPP = (\sum_{i=1}^{n} C_{i}^2)^{-1}$ , onde $C_{i}$ é a percentagem de cadeiras do partido $i$\footnote{O NEPP foi preferido ao invés do índice de Fracionalização de Rae porque, ao contrário deste, não possui um teto no valor máximo.}. Num congresso mais fragmentado, é esperado que presidentes incluam mais partidos em seus gabinetes para explorar problemas coordenativos entre estes, conforme estipula a Hipótese 2. As outras duas variáveis captam o efeito da distribuição de preferências ideológicas, que afetam o ambiente de negociações e determina os custos de cooperação. Para gerá-las, cada partido com mais de 5\% de cadeiras no congresso de cada país em cada ano foi codificado numa escala que vai de 1 a 5, da esquerda à direita; na sequência, esses \textit{scores} foram ponderados pela percentagem de cadeiras dos partidos e a variável resultante foi centrada. As informações sobre as posições dos partidos foram extraídas de três fontes diferentes, dando origem a três versões de cada variável\footnote{As fontes são descritas no Apêndice A.}. \textit{Polarização no congresso} é simplesmente o desvio-padrão da distribuição dessas preferências em cada país-ano e mensura o grau de polarização no congresso. \textit{Extremismo do presidente}, por sua vez, é igual a posição do partido do presidente menos a média das posições de todos os partidos ao quadrado, i. e., $(p_{1}~ -~\bar{p})^2$, dividido pelo somatório da posição de cada partido menos aquela mesma média ao quadrado, i. e., \textit{Extremismo} $= \frac{(p_{1}~ -~\bar{p})^2}{\sum_{i=2}^{n-1} (p_{i} ~ - ~ \bar{p})^2}$, onde $p_{i}$ é a posição ideológica do partido $i$ e $i = 1$ indica o partido do presidente. Este procedimento é necessário para tornar comparáveis os \textit{scores} entre países e anos, já que a distância do presidente em relação à média do congresso depende da posição de todos os partidos: mantendo a distância do presidente igual, maior polarização o aproxima relativamente do centro; ao contrário, menor polarização o torna relativamente mais distante (Cf. CROMBEZ, \citeyear{crombez1996}).
 
Também incluo alguns controles. \textit{Ciclo eleitoral} é igual ao tempo em anos restantes de mandato dividido pela duração total do mandato presidencial e serve para mensurar o efeito do ciclo eleitoral sobre o tamanho das coalizões. Com a proximidade das eleições, partidos teriam incentivos para sair da coalizão e disputá-las de forma independente (ALTMAN, \citeyear{altman2000}; SHUGART e CAREY, \citeyear{shugart1992}). De todo o modo, este resultado também depende da popularidade do presidente, já que a decisão de abandonar a coalizão baseia-se na utilidade de ir à oposição (MARTINEZ-GALLARDO, \citeyear{martinez2012}). Infelizmente, não existem dados sobre a aprovação presidencial em todos os países analisados. Como \textit{proxy}, entretanto, utilizo o logaritmo, atrasado em um ano, da taxa de inflação anual do \textit{World Development Indicators}\footnote{Disponível em: \url{http://data.worldbank.org/data-catalog/world-development-indicators}. Acesso em: 28/01/2015.}. A expectativa do seu efeito é simples: quanto maior a inflação, menos incentivos os partidos de oposição têm para ir ao lado governista. Por fim, também controle os resultados pela percentagem de cadeiras do partido do presidente.

\subsubsection{Modelos}

Para estimar a probabilidade de surgirem coalizões sobredimensionadas $Y_{i, t}$ num país $i$ num ano $t$, o modelo básico utilizado é

\begin{equation}
Pr(Y_{i, t} = 1 \mid X_{i, t}) = \Lambda(X_{i, t}\beta + s(t) + \varepsilon_{i, t})
\end{equation}
\noindent
onde $\Lambda(\cdot)$ é a função de distribuição cumulativa logística; $X$ é a matriz que contém as variáveis independentes; e $s(\cdot)$ é um polinômio cúbico usado para controlar a dependência temporal na variável dependente (Cf. CARTER e SIGNORINO, \citeyear{carter2010}). Como apenas 6 dos 18 países presidencialistas do Continente tiveram coalizões sobredimensionadas no período, incluir efeitos-fixos removeria arbitrariamente boa parte das observações. Ainda que ajudassem a captar heterogeneidades não-observadas, os custos de incluí-los, portanto, seriam grandes. 
 
A análise que segue também não está livre de problemas de endogeneidade comuns na política comparada. Para dar maiores garantias de que os resultados não se devem a problemas como este, portanto, realizo também uma série de \textit{robust checks} -- como o uso de uma amostra reduzida e efeitos fixos, de modelos binomiais negativos e da inclusão de diversos controles.